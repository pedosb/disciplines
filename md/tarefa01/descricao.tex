\documentclass[a4paper]{article}

\usepackage{graphicx}
\usepackage[utf8]{inputenc}

\hyphenation{ve-ri-fi-car FalaBrasil}

\title{Tarefa 01}

\author{Pedro Batista - pedro@ufpa.br}

\begin{document}

\maketitle

\section{Questão 4}
Como esperado o algoritmo 1R procurou qual o atributo
que sozinho melhor representa a classe final, isto é,
qual possui o menor erro. Por observar apenas um atributo
seu resultado não foi tão bom, pois não é suficiente
para descrever o comportamento desses dados.

O algoritmo ID3, leva em consideração quantos atributos
forem necessários, neste caso, ele escolheu Situação
Financeira e Inteligencia para classificar nossa variável
classe, e este se mostrou o melhor resultado.

Em pesquisas, o algoritmos C5.0, se mostrou uma extensão
do C4.5, este ultimo é um algoritmo livre~\cite{data}, já o C5.0, não 
está disponível livremente e não é descrito em livros, então
não foi possível usa-lo. Porém é conhecido que o C5.0 tem
poucas melhorias em relação ao C4.5, melhorando principalmente
o tempo de processamento. A ideia do C4.5, é
tentar prever o erro de cada folha de uma árvore, e o erro se essas
forem trocadas por apenas uma. Outra possibilidade é substituir 
um nó por toda sua subárvore abaixo. Muitas vezes
além de melhorar os resultados esse algoritmo melhora também
o tempo de processamento. Nesse conjunto de dados, não houve melhorias
em relação a árvore encontrada no algoritmo ID3.


\bibliographystyle{plain}
\bibliography{bib.bib} 

\end{document}
